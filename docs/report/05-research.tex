\chapter{Исследовательский раздел}
В данном разделе проводится оценка временных затрат для разработанного комбинированного алгоритма в сравнении с пошаговым и событийным алгоритмами. Также анализируются временные затраты на моделирование рассматриваемой в рамках данной работы системы массового обслуживания (МФЦ).


Ключевым недостатком пошагового алгоритма в сравнении с комбинированным является невозможность пропустить интервалы времени, не содержащие никаких событий. Как следствие, пошаговый алгоритм плохо адаптирован под квазисинхронное распределение событий, рассмотренное в аналитическом разделе. Событийный алгоритм, в свою очередь, теряет свою временную эффективность при росте очереди событий, а также при увеличении числа обращений к ней ввиду не константной сложности вставки событий.

Для измерения времени выполнения программного обеспечения используется устройство MacBook Air M1, 8-core GPU, 16 ГБ.

Для повышения точности результатов (ввиду наличия фактора случайности в процессе работы программного обеспечения) для каждого набора данных производится 100 прогонов алгоритмов с получением в итоге среднего значения.

\section{Сравнение алгоритмов}

\textbf{Квазисинхронное распределение событий}

Предлагается сравнить временные затраты пошагового, событийного и комбинированного алгоритмов на системе, состоящей из одного генератора и одного процессора. Продолжительность генерации заявки генерируется согласно равномерному квазисинхронному распределению с длиной пикового интервала в 100 тиков, минимальным и максимальным значением в 1 и 2 тика соответственно. Продолжительность обработки заявки генерируется в соответствии с равномерным распределением с минимальным и максимальным значением в 1 и 2 тика соответственно.
Моделируется работа системы в течение 100000 тиков таймера модельного времени. При этом для пикового распределения изменяется длина непикового интервала времени (интервала между пиками). Для комбинированного алгоритма размер нулевого уровня выбран в соответствии с длиной пикового интервала (100 тиков), а размер ненулевого уровня выбран равным 10. Шаг таймера модельного времени для пошагового алгоритма составляет 1 тик.

На рисунке \ref{img:graph1} представлен график сравнения временных затрат алгоритмов в зависимости от длины непиковых интервалов.

\begin{figure}[H]
	\centering
	\begin{tikzpicture}
		\begin{axis}[
			width=15cm,
			height=10cm,
			grid=major,
			grid style=dashed,
			axis lines=left,
			ylabel={Время работы, мс},
			xlabel={Длина непикового интервала},
			xmin=0, ymin=0,
			xmax=3000, ymax=500,
			xtick={0,500,...,3000},
			legend pos=north west
			]
			\addplot[color=red] coordinates {
(0, 459)
(100, 222)
(200, 148)
(300, 114)
(400, 89)
(500, 74)
(600, 63)
(700, 55)
(800, 49)
(900, 45)
(1000, 40)
(1100, 37)
(1200, 34)
(1300, 32)
(1400, 29)
(1500, 28)
(1600, 26)
(1700, 25)
(1800, 23)
(1900, 22)
(2000, 21)
(2100, 20)
(2200, 19)
(2300, 18)
(2400, 17)
(2500, 17)
(2600, 16)
(2700, 15)
(2800, 15)
(2900, 15)
(3000, 14)
			};
			\addplot[color=blue] coordinates {
(0, 281)
(100, 185)
(200, 152)
(300, 139)
(400, 130)
(500, 123)
(600, 119)
(700, 116)
(800, 115)
(900, 116)
(1000, 115)
(1100, 114)
(1200, 110)
(1300, 109)
(1400, 110)
(1500, 110)
(1600, 107)
(1700, 106)
(1800, 107)
(1900, 103)
(2000, 102)
(2100, 101)
(2200, 100)
(2300, 100)
(2400, 100)
(2500, 99)
(2600, 99)
(2700, 99)
(2800, 101)
(2900, 100)
(3000, 101)
			};
			\addplot[color=green] coordinates {
(0, 381)
(100, 169)
(200, 112)
(300, 84)
(400, 67)
(500, 56)
(600, 48)
(700, 42)
(800, 37)
(900, 33)
(1000, 30)
(1100, 28)
(1200, 25)
(1300, 24)
(1400, 22)
(1500, 21)
(1600, 19)
(1700, 18)
(1800, 17)
(1900, 16)
(2000, 16)
(2100, 15)
(2200, 14)
(2300, 14)
(2400, 13)
(2500, 13)
(2600, 12)
(2700, 12)
(2800, 11)
(2900, 11)
(3000, 11)
			};
			\legend{\text{Комбинированный}, \text{Пошаговый}, \text{Событийный}}
		\end{axis}
	\end{tikzpicture}
	\caption{Время работы комбинированного и пошагового алгоритмов в зависимости от длины непиковых интервалов}
	\label{img:graph1}
\end{figure}

Проанализируем полученный результат. При малых (0---200 тиков) размерах непиковых интервалов временные затраты комбинированного алгоритма превышают временные затраты пошагового и событийного алгоритмов. Это связано с тем, что при данных величинах непикового интервала распределение событий стремится к равномерному, т.е. не квазисинхронному. Чем меньше длина непикового интервала, тем меньше промежутков, не насыщенных событиями, сооответственно, тем меньше теряет свою эффективность пошаговый алгоритм. Однако с увеличением длины непикового интервала на временной оси появляется все больше промежутков, не содержащих (или содержащих значительно меньше событий, чем пиковые интервалы), вследствие чего пошаговый алгоритм совершает <<холостые>> обходы блоков системы. Комбинированный алгоритм же ввиду особенностей часовой структуры имеет возможность пропускать интервалы, не содержащие событий, что сокращает общее число операций. Подбор длины нулевого уровня в соответствии с длиной пикового интервала позволяет за один обход нулевого уровня обработать весь пиковый интервал целиком без необходимости повторных возвратов. Таким образом, временные затраты комбинированного алгоритма в сравнении с пошаговым будут меньше при увеличении длины непиковых интервалов.
Событийный алгоритм однако показывает лучшие результаты в связи с тем, что рассмотренная система не нагружена событиями из-за малого количества блоков (1 генератор и 1 процессор). 


\textbf{Количество блоков в системе}

Предлагается сравнить временные затраты пошагового, событийного и комбинированного алгоритмов на системе, состоящей из одного генератора и изменяющегося числа процессоров. Продолжительность генерации заявки генерируется согласно равномерному распределению с минимальным и максимальным значением в 1 и 2 тика соответственно. Продолжительность обработки заявки генерируется в соответствии с равномерным распределением с минимальным и максимальным значением в 15 и 16 тиков соответственно. Соответственно, интенсивность генерации зявок превышает интенсивность их обработки. Моделируется работа системы в течение 100000 тиков таймера модельного времени. При этом изменяется количество процессоров в системе для увеличения количества событий в системе. Для комбинированного алгоритма размер нулевого уровня выбран равным 100 тиков, а размер ненулевого уровня выбран равным 10. Шаг таймера модельного времени для пошагового алгоритма составляет 1 тик.

На рисунке \ref{img:graph2} представлен график сравнения временных затрат алгоритмов в зависимости от количества процессоров в системе.

\begin{figure}[H]
	\centering
	\begin{tikzpicture}
		\begin{axis}[
			width=15cm,
			height=10cm,
			grid=major,
			grid style=dashed,
			axis lines=left,
			ylabel={Время работы, мс},
			xlabel={Количество процессоров в системе},
			xmin=0, ymin=0,
			xmax=30, ymax=2500,
			xtick={0,5,...,30},
			legend pos=north west
			]
			\addplot[color=red] coordinates {
(0, 206)
(1, 416)
(2, 470)
(3, 508)
(4, 556)
(5, 582)
(6, 618)
(7, 639)
(8, 663)
(9, 700)
(10, 720)
(11, 731)
(12, 746)
(13, 766)
(14, 784)
(15, 804)
(16, 815)
(17, 847)
(18, 886)
(19, 900)
(20, 925)
(21, 958)
(22, 985)
(23, 1017)
(24, 1028)
(25, 1039)
(26, 1049)
(27, 1049)
(28, 1072)
(29, 1088)
(30, 1101)
			};
			\addplot[color=blue] coordinates {
(0, 139)
(1, 380)
(2, 446)
(3, 501)
(4, 535)
(5, 603)
(6, 685)
(7, 719)
(8, 767)
(9, 783)
(10, 800)
(11, 850)
(12, 915)
(13, 992)
(14, 1065)
(15, 1147)
(16, 1193)
(17, 1268)
(18, 1327)
(19, 1404)
(20, 1469)
(21, 1538)
(22, 1614)
(23, 1667)
(24, 1732)
(25, 1835)
(26, 1941)
(27, 1978)
(28, 2009)
(29, 2057)
(30, 2122)
			};
			\addplot[color=green] coordinates {
(0, 120)
(1, 359)
(2, 435)
(3, 524)
(4, 594)
(5, 613)
(6, 670)
(7, 749)
(8, 807)
(9, 821)
(10, 878)
(11, 876)
(12, 897)
(13, 920)
(14, 939)
(15, 978)
(16, 984)
(17, 999)
(18, 1023)
(19, 1044)
(20, 1065)
(21, 1085)
(22, 1112)
(23, 1126)
(24, 1155)
(25, 1177)
(26, 1190)
(27, 1216)
(28, 1242)
(29, 1254)
(30, 1281)
			};
			\legend{\text{Комбинированный}, \text{Пошаговый}, \text{Событийный}}
		\end{axis}
	\end{tikzpicture}
	\caption{Время работы алгоритмов в зависимости от количества процессоров в системе}
	\label{img:graph2}
\end{figure}

Проанализируем полученные результаты. При малом (0---3) количестве процессоров в моделируемой системе комбинированный алгоритм уступает пошаговому и событийному по временным затратам. Это связано с тем, что очередь событийного алгоритма при таких данных не заполнена событиями. А с учетом рассматриваемого распределения для генерации заявок (равномерное от 1 до 2 тиков) пошаговый алгоритм практически на каждом тике находит событие и может быть даже эффективнее событийного. Однако с ростом количества процессоров для событийного алгоритма увеличивается количество одновременных событий в очереди и обращений к очереди в целом, а с учетом нелинейной сложности операций вставки в очередь временные затраты событийного алгоритма выше, чем временные затраты комбинированного. Ввиду обхода всего списка блоков системы на каждой итерации временные затраты пошагового алгоритма также становятся выше временных затрат комбинированного алгоритма.

\section{Моделирование МФЦ}
Проведем моделирование МФЦ, рассмотренной в конструкторском разделе. Длина нулевого уровня комбинированного алгоритма подобрана в соответствии с заявленной длиной пикового интервала (10800 тиков). Рассмотрим временные затраты алгоритмов в зависимости от числа окон обслуживания в МФЦ.


\begin{figure}[H]
	\centering
	\begin{tikzpicture}
		\begin{axis}[
			width=15cm,
			height=10cm,
			grid=major,
			grid style=dashed,
			axis lines=left,
			ylabel={Время работы, мс},
			xlabel={Количество окон обслуживания},
			xmin=0, ymin=0,
			xmax=40, ymax=200,
			xtick={0,5,...,40},
			legend pos=north west
			]
			\addplot[color=red] coordinates {
(0, 55)
(1, 59)
(2, 63)
(3, 67)
(4, 70)
(5, 73)
(6, 77)
(7, 80)
(8, 83)
(9, 86)
(10, 89)
(11, 92)
(12, 94)
(13, 97)
(14, 98)
(15, 99)
(16, 100)
(17, 102)
(18, 102)
(19, 103)
(20, 104)
(21, 104)
(22, 105)
(23, 107)
(24, 107)
(25, 108)
(26, 109)
(27, 110)
(28, 111)
(29, 111)
(30, 114)
(31, 114)
(32, 114)
(33, 115)
(34, 115)
(35, 116)
(36, 118)
(37, 118)
(38, 119)
(39, 119)
(40, 120)
			};
			\addplot[color=green] coordinates {
(0, 34)
(1, 41)
(2, 48)
(3, 55)
(4, 60)
(5, 64)
(6, 70)
(7, 77)
(8, 84)
(9, 88)
(10, 92)
(11, 96)
(12, 100)
(13, 106)
(14, 105)
(15, 106)
(16, 107)
(17, 109)
(18, 113)
(19, 114)
(20, 115)
(21, 116)
(22, 119)
(23, 119)
(24, 118)
(25, 121)
(26, 123)
(27, 122)
(28, 122)
(29, 123)
(30, 123)
(31, 124)
(32, 126)
(33, 126)
(34, 127)
(35, 128)
(36, 129)
(37, 129)
(38, 130)
(39, 131)
(40, 132)
			};
			\legend{\text{Комбинированный}, \text{Событийный}}
		\end{axis}
	\end{tikzpicture}
	\caption{Время работы комбинированного и событийного алгоритмов в зависимости от количества окон обслуживания}
	\label{img:graph3}
\end{figure}


\begin{figure}[H]
	\centering
	\begin{tikzpicture}
		\begin{axis}[
			width=15cm,
			height=10cm,
			grid=major,
			grid style=dashed,
			axis lines=left,
			ylabel={Время работы, мс},
			xlabel={Количество окон обслуживания},
			xmin=0, ymin=0,
			xmax=40, ymax=7000,
			xtick={0,5,...,40},
			legend pos=north west
			]
			\addplot[color=red] coordinates {
(0, 55)
(1, 59)
(2, 63)
(3, 67)
(4, 70)
(5, 73)
(6, 77)
(7, 80)
(8, 83)
(9, 86)
(10, 89)
(11, 92)
(12, 94)
(13, 97)
(14, 98)
(15, 99)
(16, 100)
(17, 102)
(18, 102)
(19, 103)
(20, 104)
(21, 104)
(22, 105)
(23, 107)
(24, 107)
(25, 108)
(26, 109)
(27, 110)
(28, 111)
(29, 111)
(30, 114)
(31, 114)
(32, 114)
(33, 115)
(34, 115)
(35, 116)
(36, 118)
(37, 118)
(38, 119)
(39, 119)
(40, 120)
			};
			\addplot[color=blue] coordinates {
(0, 2289)
(1, 2337)
(2, 2396)
(3, 2452)
(4, 2517)
(5, 2584)
(6, 2686)
(7, 2693)
(8, 2756)
(9, 2827)
(10, 2891)
(11, 2957)
(12, 3028)
(13, 3099)
(14, 3212)
(15, 3342)
(16, 3478)
(17, 3598)
(18, 3731)
(19, 3864)
(20, 3988)
(21, 4118)
(22, 4245)
(23, 4371)
(24, 4504)
(25, 4627)
(26, 4763)
(27, 4889)
(28, 5014)
(29, 5158)
(30, 5285)
(31, 5408)
(32, 5535)
(33, 5667)
(34, 5759)
(35, 5973)
(36, 6083)
(37, 6219)
(38, 6350)
(39, 6484)
(40, 6620)
			};
			\legend{\text{Комбинированный}, \text{Пошаговый}}
		\end{axis}
	\end{tikzpicture}
	\caption{Время работы комбинированного и пошагового алгоритмов в зависимости от количества окон обслуживания}
	\label{img:graph4}
\end{figure}

\begin{figure}[H]
	\centering
\begin{tikzpicture}
	\begin{axis}[
		ybar,
		width=15cm,
		height=10cm,
		xtick={1,2,3,4,5,6,7},
		xticklabels={пн,вт,ср,чт,пт,сб,вс},
		ylabel={Кол-во посетителей, чел},
		legend style={
			at={(0.5,-0.25)},
			anchor=south,
			legend columns=-1
		}
		]
\addplot coordinates {
	(1, 47)
	(2, 117)
	(3, 98)
	(4, 35)
	(5, 37)
	(6, 31)
	(7, 23)
};
\addplot coordinates {
	(1, 189)
	(2, 548)
	(3, 461)
	(4, 180)
	(5, 163)
	(6, 146)
	(7, 131)
};
\addplot coordinates {
	(1, 34)
	(2, 116)
	(3, 85)
	(4, 45)
	(5, 38)
	(6, 38)
	(7, 17)
};
\addplot coordinates {
	(1, 25)
	(2, 109)
	(3, 104)
	(4, 46)
	(5, 35)
	(6, 36)
	(7, 22)
};
		\legend{08.00-11.00, 11.00-14.00, 14.00-17.00, 17.00-20.00}
	\end{axis}
\end{tikzpicture}
	\caption{Время работы комбинированного и пошагового алгоритмов в зависимости от количества окон обслуживания}
\label{img:graph5}
\end{figure}


Временные затраты событийного алгоритма оказались в среднем в 1.2 раза выше временных затрат комбинированного алгоритма при количестве окон > 7.
Временные затраты пошагового алгоритма оказались в среднем в 35 раз выше временных затрат комбинированного алгоритма.

\section{Выводы}
В рамках данного раздела было проведено исследование характеристик разработанного комбинированного алгоритма продвижения модельного времени. Также был проведен сравнительный анализ комбинированного алгоритма с пошаговым и событийным.