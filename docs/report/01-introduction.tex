\chapter*{ВВЕДЕНИЕ}
\addcontentsline{toc}{chapter}{ВВЕДЕНИЕ}

Имитационное моделирование --- это универсальный способ цифрового
представления реальной системы при помощи средств компьютерной техники, вычислительных алгоритмов и технологий программирования. Потребность в имитационном моделировании возникает в связи с дорогими и/или невозможными исследованиями реальных систем. 

На сегодняшний день выделяют три вида имитационного моделирования: агентное моделирование, дискретно-событийное моделирование и системную динамику.
Наиболее развитым подходом среди перечисленных является дискретно-событийное моделирование \cite{economic_system}. При таком подходе не учитывается непрерывная природа событий, и рассматриваются только основные изменения состояния моделируемой системы, а именно события, вызывающие эти изменения. 

Дискретно-событийное моделирование применяется во многих областях, где необходимо анализировать и оптимизировать производительность системы. К областям применения можно отнести производственные системы, логистику, финансовую аналитику, бизнес-процессы, транспортные системы и другие. Дискретно-событийное моделирование может помочь оптимизировать использование ресурсов, повысить эффективность производства, определить оптимальные стратегии управления запасами, маршрутизации транспорта, инвестирования и управления рисками, оптимизировать бизнес-процессы, использование дорожных сетей и многое другое.

Таким образом, имитационное, а в частности дискретно-событийное, моделирование --- мощный инструмент для анализа и оптимизации производительности систем, имеющий применение во множестве различных областей.